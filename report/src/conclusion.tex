\section{Conclusion}
Cloud services run many of the modern applications we use today, and are expected to be constantly available.
This is not always the case, as services outages happen relatively frequently.
When a service goes down, other services that depend on it may also fail.
Therefore, prevention is key, and understanding how these outages happen is the first step towards preventing them.
In this paper, we have conducted a study of public cloud provider reports of outages during the 2018 calendar year.
We developed a framework for classifying and analyzing these reports, and used it to examine the collected data.
We identified the challenges associated with the gathering and analysis of failure data, and extracted patterns relating to the failure of cloud services.
Some of the data we generated and analyzed in this study has been used for a section of a paper submitted to a top conference.

Future work in this area could address some of the limitations and concerns discussed in \autoref{sec:threats to validity}.
In particular, the classification process could be improved, perhaps via language processing algorithms.
This would eliminate the need to classify events manually, reducing the probability of erroneous classification.
Furthermore, data limitations could be addressed by conducting a synthesis of all data from various sources, such as news reports, provider reports, and user reports.
Data from multiple sources for a particular outage could be correlated using timestamps, or other identifying features.
This would provide a more objective view of the data, and would result in a much larger dataset.
