\section{Self-reflection}
This was the first opportunity I had to conduct a `proper' research project, and as such benefited me greatly.
I learned how to clean a large dataset, and how to conduct a scientific analysis of that dataset, including data visualisation.
I learned how to use software tools such as Pandas, which is an industry standard for data analysis, and Plotnine for data plotting -- these skills will be useful in any future projects that use large sets of data.
I also learned how to compile the process and results into a research paper, and how to look for and discuss related articles.
The possibility of contributing to a paper submitted to a top conference was a great motivator.
Overall, the project helped me develop into a more independent researcher.

The largest period of time was spent on the preparation of the dataset; that is, on cleaning and labelling the dataset for analysis.
The manual classification of events was especially time-consuming.
The second longest amount of time was spent on analysis of data, particularly on selecting the best aspects to analyze and on creating good visualisations of the results.
The rest of the time was spent on researching, and on writing the report.
