\section{Problem statement}
The main question we try to answer is: how do cloud services fail?
To this end, we use provider-reported data from public status pages.
Finding an answer to this question is important, because knowing how services fail can pave the way towards partial or total prevention of such failures.
This would allow cloud providers to come closer to their goal of constant availability, and would make cloud services more dependable.

Our results show only a part of the picture, and further research is necessary to clarify our observations, as well as those of other research in the area.
One related question is: how can we create a more complete view of cloud service failures?
For example, how could a synthesis of the various data sources used in the work mentioned in \autoref{sec:related work} and the data source used in this study help with failure analysis?
Or, would it perhaps be useful to create a repository similar to the Failure Trace Archive \cite{javadi2013} for service failure data, and how could such a repository be implemented?
More questions could be asked about the analysis process: how can we improve the accuracy of failure classifications based on textual descriptions?
What other classification frameworks would allow us to extract the most information from failure reports?
These and other questions can be explored in future work.
