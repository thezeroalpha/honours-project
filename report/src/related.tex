\section{Related work}
Having presented and discussed our results, we examine previous work done in the field.

One similar study is that of Gunawi et al. \cite{gunawi2016}, who analyze the availability of popular cloud services during a time period of seven years.
They examine services for online chat (e.g. WhatsApp), e-commerce (e.g. Amazon.com), email (e.g. GMail), SaaS (e.g. Salesforce), video streaming (e.g. Netflix), and others.
They collect data from news headlines, and from the providers' public post-mortem reports (published after the failure event).
Similarly to our study, they manually tag the outages with metadata.
This metadata describes the root causes, impacts, fixes, downtime, type (planned or unplanned), and service scope of each outage.
Due to the nature of their dataset, they are able to extract more detail in some metadata categories, such as root causes.
They find that almost 50\% of the services they analyze experience an average of three or more outages per year, and 25 services do not reach 99.9\% uptime.
These outages do not decrease as the service matures; rather, as a service matures, it becomes more popular, and thus need to handle more users and increases in complexity.
Perhaps complementary to our findings regarding root causes of outages, fixing software is a major fix procedure category in their results, accounting for 22\% of all outages.
Their analysis of root causes shows that to ensure that cloud services do not have a single point of failure, perfection is required along the whole failure recovery chain.
They also conduct a detailed analysis of root causes of outages.

Another related study focuses on popular cloud systems, particularly on those generally not seen by end users \cite{gunawi2014}.
These include Hadoop MapReduce, Hadoop File System, HBase, Cassandra, ZooKeeper, and Flume.
The aim of the study is to characterise the bugs present in cloud systems.
Gunawi et al. collect data from issue repositories maintained by the development projects of the aforementioned systems.
The issues are generally submitted by the developer or user community of the system.
The authors select issues submitted over a period of three years, and analyze patches and developer responses for the issues.
They conduct a manual classification of the issues for each service, applying metadata labels related to the aspect of the issue (reliability, performance, availability, etc.), hardware type, failure mode (stop, corrupt, or ``limp''), and bug scope (single machine, multiple machines, or a whole cluster).
They find that 8\% of bugs are unique to cloud systems, and that even though there are software measures in place, 13\% of issues are still caused by hardware failures.
Their results also suggest that cascades of failures can occur in subtle ways from a ``killer bug'', which is perhaps similar to how failures can cascade across services in our analysis.
Finally, the largest category of software bugs in their results consists of those that are logic-specific, which complements our findings.
