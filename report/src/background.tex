\section{Background information}
\paragraph{Service architecture}
There are three major cloud service providers: Amazon (Amazon Web Services, AWS), Microsoft (Azure), and Google (Google Cloud Platform, GCP) \cite{dignan2020, jones2020}.
A provider makes available a number of services to its customers, and the customers are free to choose which services they want to use for their application.
A service is not tied to one physical location, but can run in any of the regions offered by the provider, and can potentially span multiple regions.
The geographical location of the service can often also be configured by the client.
A particular service may depend on other services offered by the provider.
For example, if a client decides to use AWS Elastic Beanstalk (EBS), they will also be relying on a number of other AWS resources, such as the Amazon Elastic Compute Cloud (EC2) and Amazon Simple Storage Service (S3) \cite{awsFaq}.
Each of these service dependencies may itself have other dependencies, which results in a complex dependency graph, where the failure of a single service may result in the outage of multiple other services.
Therefore, service availability is of utmost importance for cloud providers.

\paragraph{Terms \& definitions}
We define a service \textit{failure (event)} or \textit{outage} as a period of time where the service is not functioning as expected.
The \textit{expected functionality} of a service is defined both explicitly in Service Level Agreements, and implicitly by the assumptions of clients and end users.
During a failure, a service may become \textit{unavailable}, which occurs when one or more users cannot access the service.
However, many services utilize fault-tolerance techniques, and thus do not fail entirely, but instead experience \textit{performance degradation}.
This is when there is an increase in the latency of the service, or in the error rates of requests made to the service.
A service may fail in a single region, or in multiple regions.
In the case of AWS, a failure may be restricted to a single availability zone inside of a region.
Due to the interdependence of services, the failure of one service can cause failures of upstream services, resulting in a \textit{multi-service} failure.
A failure can have a varying \textit{area of effect}, which we define as the affected users and the geographical range of the failure (for example, some users in a single region).
We define the \textit{impact level} of an outage as the number of affected services, and the geographical range of the outage (for example, one service in a single region).
The \textit{duration} of an failure is the period of time between the start of a reported failure and the end of the failure.
